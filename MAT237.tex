\documentclass[11pt]{article}

% Libraries.
\usepackage{amsmath}
\usepackage{amssymb}
\usepackage{pgfplots}
\usepackage{graphicx}
\usepackage{enumitem}
\usepackage{hyperref}
\usepackage{fancyhdr}
\usepackage{perpage}
\usepackage{float}

% Property settings.
\MakePerPage{footnote}
\pagestyle{fancy}
\lhead{Notes by YW, TX}

% Commands
\newcommand{\ti}[1]{\textit{#1}}
\newcommand{\tb}[1]{\textbf{#1}}
\newcommand{\mb}[1]{\mathbb{#1}}
\newcommand{\real}[0]{\mathbb{R}}
\newcommand{\under}[1]{\underline{#1}}
\newcommand{\proof}[0]{\textit{\underline{proof:} }}
\newcommand{\func}[3]{\tb{#1}: {#2} \rightarrow {#3} }

% Attr.
\title{MAT237 Multivariable Calculus \\ Lecture Notes}
\author{Yuchen Wang, Tingfeng Xia}
\date{\today}

\begin{document}
	\maketitle
	\tableofcontents
	\newpage
\section{Critical Points}
\paragraph{Definition} A symmetric $n \times n$ matrix A is
\begin{enumerate}
	\item \tb{positive definite} if $\tb{x}^T A \tb{x} > 0$ for all $x \in \real^n \symbol{92} \{\tb{0}\}$
	\item \tb{nonnegative definite} if $\tb{x}^T A \tb{x} \geq 0$ for all $x \in \real^n$
\end{enumerate}
In addition, we say that A is
\begin{enumerate}
	\item \tb{negative definite} if -A is positive definite
	\item \tb{nonpositive definite} if -A is nonnegative definite
\end{enumerate}
A matrix A is \tb{indefinite} if none of the above holds. Equivalently, A is indefinite if there exist $\tb{x, y}\in \real$ such that $\tb{x}^TA\tb{x} < 0 < \tb{y}^TA\tb{y}$
\paragraph{Theorem 1} Assume that A is a symmetric matrix. Then \newline
\begin{enumerate}
	\item A is positive definite $\iff$ all its eigenvalues are positive \newline
$\iff \exists \lambda_1 > 0$ such that $\tb{x}^TA\tb{x} \geq \lambda_1|\tb{x}|^2$ for all $\tb{x} \in \real^n $
	\item A is nonnegative definite $\iff$ all its eigenvalues are nonnegative \newline
	\item A is indefinite $\iff$ A has both positive and negative eigenvalues
\end{enumerate}
\paragraph{Remark} If A is a symmetric matrix then \newline
The smallest eigenvalue of A = $\min_{\{\tb{u}\in \real^n: |\tb{u}| = 1\}} \tb{u}^TA\tb{u}$
\paragraph{Theorem 2} For the matrix $A = \begin{pmatrix}
	\alpha & \beta \\
	\beta & \gamma 
\end{pmatrix}$,
\begin{enumerate}
	\item if $det A < 0,$ then A is indefinite
	\item if $det A > 0$, then
	\subitem if $\alpha > 0$ then A is positive definite
	\subitem if $\alpha < 0$ then A is negative definite
	\item if $det A = 0$ then at least one eigenvalue equals zero.
\end{enumerate}
\paragraph{Definition} A critical point $\tb{a}$ of $C^2$ function $\tb{f}$ is \under{degenerate} if det$(D_\tb{H}(\tb{a})) = 0$
\paragraph{Theorem 3 - first derivative test} If $\tb{f}: S \in \real^n \rightarrow \real$ is differentiable, then every local extremum is a critical point.
\paragraph{Theorem 4 - second derivative test}
\begin{enumerate}
	\item If $f: S \rightarrow \real$ is $C^2$ and \tb{a} is a local minimum point for $f$, then \tb{a} is a critical point of $f$ and $H(\tb{a})$ is nonnegative definite.
	\item If \tb{a} is a critical point and $H(\tb{a})$ is positive definite, then \tb{a} is a local minimum point.
\end{enumerate}
\paragraph{Corollary} Assume that $f$ is $C^2$ and $\nabla f(\tb{a}) = \tb{0}$
\begin{enumerate}
	\item If H(a) is positive definite, then a is a local min;
	\item If H(a) is negative definite, then a is a local max;
	\item If H(a) is indefinite, then a is a saddle point;
	\item If none of the above hold, then we cannot determine the character of the critical point without further thought.
\end{enumerate} 
\section{The Implicit Function Theorem}
Assume that S is an open subset of $\real^{n+k}$ and that $F: S \rightarrow \real^k$ is a function of class $C^1$. Assume also that $(\tb{a}, \tb{b})$ is a point in S such that $\tb{F(a, b)=0}$ and det$D_{\tb{y}}\tb{F(a, b)} \neq 0$ \newline
1. Then there exists $r_0, r_1 > 0$ such that for every $\tb{x} \in \real^n$ such that $|\tb{x} - \tb{a}| < r_0$, there exists a unique $\tb{y} \in \real^k$ such that $|\tb{y} - \tb{b}| < r_1$
	$$\tb{F(x, y) = 0} (1)$$
	In other words, equation (1) implicitly defines a function $\tb{y = f(x)}$ for $x \in \real^n$ near \tb{a}, with \tb{y = f(x)} close to \tb{b}. Note in particular that \tb{b = f(a)}. \newline
2. Moreover, the function $\tb{f}: B(r_0,\tb{a}) \rightarrow B(r_1, \tb{b}) \subset \real^k$ from part (1) above is of class $C^1$, and its derivatives may be determined by differentiating the identity $$\tb{F(x,f(x)) = 0}$$ and solving to find the partial derivatives of \tb{f}.
\paragraph{Remark} $$D\tb{f(a)} = -[D_\tb{y}\tb{F(a, b)}]^{-1}D_\tb{x}\tb{F(a, b)}$$
\section{The Inverse Function Theorem} Let U and V be open sets in $\real^n$, and assume that $\func{f}{U}{V}$ is a mapping of class $C^1$. \newline
Assume that \tb{a} $\in U$ is a point such that $D\tb{f(a)}$ is invertible. \newline
and let $\tb{b} := \tb{f(a)}$. Then there exist open sets $M \subset U$ and $N \subset V$ such that
\begin{enumerate}
	\item $\tb{a} \in M$ and $\tb{b} \in N$
	\item $\tb{f}$ is one-to-one from M onto N (hence invertible), and
	\item the inverse function $f^{-1}: N \rightarrow M$ is of class $C^1$
\end{enumerate}
Moreover, if $x \in M$ and $y = \tb{f(x)}\in N$, then $$D(\tb{f}^{-1})(\tb{y}) = [D\tb{f(x)}]^{-1}$$
In particular, $$D(\tb{f}^{-1})(\tb{b}) = [D\tb{f(a)}]^{-1}$$

\section{Theorems of 1-D Integral Calculus}
\paragraph{Lemma: Refined partitions give better approximations} Let $P$ be some partition over an interval and let $P'$ be a refinement of $P$, then
\begin{equation*}
    LS_{P'}f \geq LS_{P}f \wedge US_{P'} \leq US_{P}f
\end{equation*}
Where LS and US stands for lower sum and upper sum respectively.

\paragraph{Lemma: Lower sum is always less then or equal to upper sum} If $P$ and $Q$ are any partitions of $[a,b]$, then $LS_Pf \leq US_Qf$. The essence of this proof is to consider the common refinement of these two partitions.

\paragraph{Lemma. $\epsilon-\delta$ definition of integrability} If $f$ is a bounded function on $[a,b]$, the following conditions are equivalent:
\begin{enumerate}
    \item $f$ is integrable on $[a,b]$
    \item $\forall \epsilon > 0, \exists P$ of $[a,b]$ such that $US_Pf - LS_Pf < \epsilon$
\end{enumerate}

\paragraph{Theorem:  Integration is ``Linear"}
\begin{enumerate}
    \item Suppose $a < b<c$. If $f$ is integrable on $[a,b]$ and on $[b,c]$, then $f$ is integrable on $[a,c]$, further more
    \begin{equation*}
        \int_a^c f(x)dx = \int_a^b f(x)dx + \int_b^c f(x)dx
    \end{equation*}
    \item If $f$ and $g$ are integrable on $[a,b]$, then so is $f+g$, further more
    \begin{equation*}
        \int_a^b [f(x) + g(x)]dx = \int_a^b f(x)dx + \int_a^b g(x)dx
    \end{equation*}
\end{enumerate}

\paragraph{Theorem.} Suppose $f$ is integrable on $[a,b]$.
\begin{enumerate}
    \item If $c\in \mathbb{R}$, the $cf$ is integrable on $[a,b]$, and $\int_a^b cf(x) = c\int_a^bf(x)dx$
    \item Of $[c,d] \subset [a,b]$, then $f$ is integrable on $[c,d]$.
    \item If $g$ is integrable on $[a,b]$ and $f(x) \leq g(x),\forall x \in [a,b]$, then $\int_a^b f(x)dx\leq \int_a^b g(x)dx$
    \item $|f|$ is integrable on $[a,b]$, and $|\int_a^bf(x)dx| \leq \int_a^b |f(x)|dx$
\end{enumerate}

\paragraph{Theorem: Bounded + monotone $\implies$ integrable} If $f$ is bounded and monotone on $[a,b]$, then $f$ is integrable on $[a,b]$. The proof of this uses the $\epsilon-\delta$ definition of integrability

\paragraph{Theorem: Continuous $\implies$ integrable} If $f$ is continuous on $[a,b]$, then $f$ is integrable on $[a,b]$. Note that continuous is a sufficient but not necessary condition of integrability

\paragraph{Theorem: discontinuous at only finite pts $\implies$ integrable} If $f$ is bounded on $[a,b]$ and continuous at all except finitely many points in $[a,b]$, then $f$ is integrable on $[a,b]$. A easy example of this would be any $\mathbb{R}$ function that has a hole in it.

\paragraph{Theorem: Discontinuous at only zero content $\implies$ integrable} If $f$ is bounded on $[a,b]$ and the set of points in $[a,b]$ at which $f$ is discontinuous has zero content, then $f$ is integrable on $[a,b]$.

\paragraph{Proposition.} Suppose $f$ and $g$ are integrable on $[a,b]$ and $f(x) = g(x)$ for all except finitely many points $x\in [a,b]$. Then $\int_a^bf(x)dx = \int_a^bg(x)dx$. 

\paragraph{The Fundamental Theorem Of Calculus}
\begin{enumerate}
    \item Let $f$ be an integrable function on $[a,b]$. For $x\in [a,b]$, let $F(x) = \int_a^xf(t)dt$. Then $F$ is continuous on $[a,b]$; more-over, $F'(x)$ exists and equals $f(x)$ at every $x$ at which $f$ is continuous,
    \item Let $F$ be a continuous function on $[a,b]$ that is differentiable except perhaps at finitely many points in $[a,b]$, and let $f$ be a function on $[a,b]$ that agrees with $F'$ at all points where the latter is defined. If $f$ is integrable on $[a,b]$, then $\int_a^bf(t)dt=F(b)-F(a)$
\end{enumerate}

\paragraph{Proposition.} Suppose $f$ is integrable on $[a,b]$. Given $\epsilon>0, \exists \delta > 0$ such that if $P= \{x_0,...,x_J\}$ is any partition of $[a,b]$ satisfying
\begin{equation*}
    max\{x_j-x_{j-1} | 1\leq j \leq J\} < \delta
\end{equation*}
the sums $LS_Pf$ and $US_Pf$ differ from $\int_a^bf(x)dx$ by at most $\epsilon$.

\section{Generalized Integral Calculus}
\paragraph{}

\end{document}

